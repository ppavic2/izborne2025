%%%%%%%%%%%%%%%%%%%%%%%%%%%%%%%%%%%%%%%%%%%%%%%%%%%%%%%%%%%%%%%%%%%%%%
% Problem statement
\begin{statement}[
  problempoints=100,
  timelimit=1 sekunda,
  memorylimit=512 MiB,
]{Trava}\

U mirnom kutku grada nalazi se umirovljenički dom čiji stanari vole provoditi vrijeme promatrajući travnjak ispred zgrade. Travnjak je podijeljen na $N$ segmenata, a svaki segment ima visinu trave $a_i$ milimetara, za $1 \leq i \leq N$.

Umirovljenici, zbog godina i dioptrije, ne vide baš savršeno. Kada umirovljenik s dioptrijom $k$ promatra travnjak, on ne razlikuje pojedinačne segmente unutar $k$ uzastopnih dijelova travnjaka. Formalnije, umirovljenik s dioptrijom $k$ na poziciji $i$ vidi visinu trave $\max(a_i, a_{i+1}, \ldots, a_{i+k-1})$ milimetara, za sve $1 \leq i \leq N - k + 1$, dok ostale pozicije ne promatra.

Osim toga, s vremena na vrijeme trava na nekom segmentu može narasti za jedan milimetar, čime se mijenja izgled cijelog travnjaka, a time i visina koje umirovljenici vide.

Potrebno je obraditi $Q$ upita sljedećih oblika:
\vspace{-0.7em}
\begin{itemize}
    \item \texttt{1 k} — umirovljenik s dioptrijom $k$ promatra travnjak. Odredi sumu svih visina koje on vidi.
    \item \texttt{2 i} — trava na $i$-tom segmentu naraste za jedan milimetar.
\end{itemize}

%%%%%%%%%%%%%%%%%%%%%%%%%%%%%%%%%%%%%%%%%%%%%%%%%%%%%%%%%%%%%%%%%%%%%%
% Input
\subsection*{Ulazni podaci}

U prvom retku nalazi se prirodan broj $N$ — broj segmenata travnjaka.

U drugom retku nalazi se $N$ cijelih brojeva $a_1, a_2, \ldots, a_N$ — početne visine trave.

U trećem retku nalazi se cijeli broj $Q$ — broj upita.

U idućih $Q$ redaka nalazi se po jedan upit opisan kao:
\vspace{-0.7em}
\begin{itemize}
    \item \texttt{1 k} ($1 \leq k \leq N$)
    \item \texttt{2 i} ($1 \leq i \leq N$)
\end{itemize}

%%%%%%%%%%%%%%%%%%%%%%%%%%%%%%%%%%%%%%%%%%%%%%%%%%%%%%%%%%%%%%%%%%%%%%
% Output
\subsection*{Izlazni podaci}

Za svaki upit tipa \texttt{1 k}, ispiši u zaseban redak jedan cijeli broj — sumu najvećih visina u svim prozorima duljine $k$.

%%%%%%%%%%%%%%%%%%%%%%%%%%%%%%%%%%%%%%%%%%%%%%%%%%%%%%%%%%%%%%%%%%%%%%
% Scoring
\subsection*{Bodovanje}

%%%%%%%%%	NAPISATI OGRANICENJA
%%%%%%%%%   prepraviti prirodan / cijeli broj iz ulaza po potrebi

{\renewcommand{\arraystretch}{1.4}
  \setlength{\tabcolsep}{6pt}
  \begin{tabular}{ccl}
   Podzadatak & Broj bodova & Ograničenja \\ \midrule
   	1 & 4 & $N \leq 20$ \\
    2 & 7 & $c_i = 1$ za sve $i$ i dodatno ako je $j$ šef od $i$ tada $p_j \geq p_i$. \\
    3 & 23 & Za sve $i < N$, izravan šef od $i + 1$ je $i$. \\
    4 & 13 & $N, K \leq 500$ \\
    5 & 18 & $N \leq 100$ \\
    6 & 35 & Nema dodatnih ograničenja. \\
\end{tabular}}

%%%%%%%%%%%%%%%%%%%%%%%%%%%%%%%%%%%%%%%%%%%%%%%%%%%%%%%%%%%%%%%%%%%%%%
% Examples
\subsection*{Probni primjeri}
\begin{tabularx}{\textwidth}{X'X'X}
\sampleinputs{test/trava.dummy.in.1}{test/trava.dummy.out.1} &
\sampleinputs{test/trava.dummy.in.2}{test/trava.dummy.out.2} &
\sampleinputs{test/trava.dummy.in.3}{test/trava.dummy.out.3}
\end{tabularx}

\pagebreak

\textbf{Pojašnjenje drugog probnog primjera:}\\


%%%%%%%%%%%%%%%%%%%%%%%%%%%%%%%%%%%%%%%%%%%%%%%%%%%%%%%%%%%%%%%%%%%%%%
% We're done
\end{statement}

%%% Local Variables:
%%% mode: latex
%%% mode: flyspell
%%% ispell-local-dictionary: "croatian"
%%% TeX-master: "../hio.tex"
%%% End:
