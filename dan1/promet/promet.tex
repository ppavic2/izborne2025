%%%%%%%%%%%%%%%%%%%%%%%%%%%%%%%%%%%%%%%%%%%%%%%%%%%%%%%%%%%%%%%%%%%%%%
% Problem statement
\begin{statement}[
  problempoints=100,
  timelimit=2 sekunde,
  memorylimit=1024 MiB,
]{Dvoboj}

\textit{Dvije faraonske žute linije su se pretvorile u oko...}

Mladi Jusuf ima $N$ karata u svojem špilu, poredanih s lijeva na desno od $1$ do $N$. Svaka karta ima svoju snagu koju ćemo označavati s $p_i$. Jusuf se želi pripremiti za nadolazeći turnir, pa bi htio isprobati bitke između svojih karata te izmjenjivati karte u svojem špilu raznim drugim kartama koje je dobio na poklon od djeda. Ukupno će Jusuf napraviti $Q$ upita od kojih će svaki biti jednog od sljedeća dva tipa:

\begin{itemize}
\item \texttt{1 i r} - označava upit u kojem je Jusuf kartu na poziciji $i$ zamijenio novom kartom sa snagom $r$
\item \texttt{2 l k} - Jusuf će zamisliti imaginarnu bitku s $2^k$ karata, počevši od $l$-te te završivši s $l + 2^k - 1$-tom, te zaderati se \textit{Vrijeme je za dvoboj!}. Bitka će se odvijati u $k$ koraka. U svakom koraku, Jusuf će promatrati parove susjednih karata (prvu i drugu, treću i četvrtu itd.) te usporediti njihove snage, neka su u jednom paru to $A$ i $B$. Karta s većom snagom će pobijediti, te će njezina nova snaga iznositi $|A - B|$ (kojagod karta pobijedila). Ako su karte jednake snage, bitka će biti neizvjesna te će nasumična karta pobijediti i njezina će snaga biti $0$. Karta koja je izgubila ne sudjeluje u preostalim rundama. Primijetite da nakon $k$ ovakvih koraka, ostat će točno jedna karta. Jusufa zanima njezina snaga! 
\end{itemize} 

%%%%%%%%%%%%%%%%%%%%%%%%%%%%%%%%%%%%%%%%%%%%%%%%%%%%%%%%%%%%%%%%%%%%%%
% Input
\subsection*{Ulazni podaci}

U prvom retku su prirodni brojevi $N$ i $Q$.

U sljedećem retku nalazi se $N$ brojeva $p_i$ ($0 \leq p_i \leq 10^9$) koji označavaju snage karata.

U sljedećih $Q$ redaka nalaze se opisi upita koji odgovaraju tekstu zadatka. 

Za svaki upit tipa $1$ vrijedi $1 \leq i \leq N$ te $0 \leq r \leq 10^9$.

Za svaki upit tipa $2$ vrijedi $1 \leq l \leq N$ te $1 \leq l + 2^k - 1 \leq N$.

%%%%%%%%%%%%%%%%%%%%%%%%%%%%%%%%%%%%%%%%%%%%%%%%%%%%%%%%%%%%%%%%%%%%%%
% Output
\subsection*{Izlazni podaci}

Za svaki upit tipa $2$ potrebno je ispisati snagu završne karte nakon svih $k$ koraka.

%%%%%%%%%%%%%%%%%%%%%%%%%%%%%%%%%%%%%%%%%%%%%%%%%%%%%%%%%%%%%%%%%%%%%%
% Scoring
\subsection*{Bodovanje}

U svim podzadacima vrijedi $2 \leq N \leq 200\,000$ i $1 \leq Q \leq 200\,000$.

{\renewcommand{\arraystretch}{1.4}
  \setlength{\tabcolsep}{6pt}
  \begin{tabular}{ccl}
   Podzadatak & Broj bodova & Ograničenja \\ \midrule
    1 & 11 & $N, Q \leq 1000$ \\
    2 & 13 & Za sve upite tipa $2$ vrijedi $N = 2^k$. \\
    3 & 16 & Za sve $1 \leq i \leq N$ vrijedi $p_i \leq 1$ te za sve upite tipa $1$ vrijedi $r \leq 1$.\\
    4 & 17 & Nema upita tipa $1$. \\
    5 & 43 & Nema dodatnih ograničenja. \\
\end{tabular}}

%%%%%%%%%%%%%%%%%%%%%%%%%%%%%%%%%%%%%%%%%%%%%%%%%%%%%%%%%%%%%%%%%%%%%%
% Examples
\subsection*{Probni primjeri}
\begin{tabularx}{\textwidth}{X'X'X}
\sampleinputs{test/dvoboj.dummy.in.1}{test/dvoboj.dummy.out.1} &
\sampleinputs{test/dvoboj.dummy.in.2}{test/dvoboj.dummy.out.2} &
\sampleinputs{test/dvoboj.dummy.in.3}{test/dvoboj.dummy.out.3}
\end{tabularx}

\textbf{Pojašnjenje prvog probnog primjera:}\\

U prvom upitu karte će se ovako mijenjati tijekom koraka:

\[ (4, 8, 2, 0) \rightarrow (4, 2) \rightarrow (2) \]

U trećem upitu karte će se ovako mijenjati tijekom koraka:

\[ (8, 2) \rightarrow (6) \]
  
%%%%%%%%%%%%%%%%%%%%%%%%%%%%%%%%%%%%%%%%%%%%%%%%%%%%%%%%%%%%%%%%%%%%%%
% We're done
\end{statement}

%%% Local Variables:
%%% mode: latex
%%% mode: flyspell
%%% ispell-local-dictionary: "croatian"
%%% TeX-master: "../hio.tex"
%%% End:
