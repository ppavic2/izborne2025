%%%%%%%%%%%%%%%%%%%%%%%%%%%%%%%%%%%%%%%%%%%%%%%%%%%%%%%%%%%%%%%%%%%%%%
% Problem statement
\begin{statement}[
  problempoints=100,
  timelimit=1 sekunda,
  memorylimit=1024 MiB,
]{Promet}\

Bliže se lokalni izbori!

Sve vrvi od različitih prometnih planova, a malog Ivicu zanima samo jedno pitanje, koliko će mu zanimljiv biti put od škole!

Možemo zamisliti da se Zagreb sastoji od $N$ kvartova označenih brojevima od $1$ do $N$. Između nekih parova kvartova $i$ te $j$ (gdje $i < j$) postoje jednosmjerne ulice. \textit{Prometni plan} sastoji se od nekog skupa takvih jednosmjernih ulica. 

Ivičina kuća nalazi se u kvartu $1$, a škola u kvartu $N$. Sada ga zanima, za svaki $K$ od $0$ do $N$, koliko postoji prometnih planova, tako da broj kvartova koji se nalaze na \textbf{nekom} mogućem putu od kvarta $1$ do kvarta $N$ je \textbf{točno} $K$.

Kako su ti brojevi možda jako veliki, zanima ga njihov ostatak pri dijeljenju s $P$.


%%%%%%%%%%%%%%%%%%%%%%%%%%%%%%%%%%%%%%%%%%%%%%%%%%%%%%%%%%%%%%%%%%%%%%
% Input
\subsection*{Ulazni podaci}

U prvom retku su prirodni brojevi $N$ i $P$.

%%%%%%%%%%%%%%%%%%%%%%%%%%%%%%%%%%%%%%%%%%%%%%%%%%%%%%%%%%%%%%%%%%%%%%
% Output
\subsection*{Izlazni podaci}

U jedini redak ispišite $N + 1$ brojeva gdje $i$-ti broj predstavlja broj prometnih planova s $i - 1$ bitnih kvartova modulo $P$.

%%%%%%%%%%%%%%%%%%%%%%%%%%%%%%%%%%%%%%%%%%%%%%%%%%%%%%%%%%%%%%%%%%%%%%
% Scoring
\subsection*{Bodovanje}

U svim podzadacima vrijedi $2 \leq N \leq 2\,000$ i $10^8 \leq P \leq 10^9 + 100$, $P$ je prost broj.

{\renewcommand{\arraystretch}{1.4}
  \setlength{\tabcolsep}{6pt}
  \begin{tabular}{ccl}
   Podzadatak & Broj bodova & Ograničenja \\ \midrule
   	1 & 4 & $N \leq 7$ \\
    2 & 7 & $N \leq 18$ \\
    3 & 23 & $N \leq 50$ \\
    4 & 13 & $N \leq 100$ \\
    5 & 18 & $N \leq 300$ \\
    6 & 35 & Nema dodatnih ograničenja. \\
\end{tabular}}

%%%%%%%%%%%%%%%%%%%%%%%%%%%%%%%%%%%%%%%%%%%%%%%%%%%%%%%%%%%%%%%%%%%%%%
% Examples
\subsection*{Probni primjeri}
\begin{tabularx}{\textwidth}{X'X'X}
\sampleinputs{test/promet.dummy.in.1}{test/promet.dummy.out.1} &
\sampleinputs{test/promet.dummy.in.2}{test/promet.dummy.out.2} &
\sampleinputs{test/promet.dummy.in.3}{test/promet.dummy.out.3}
\end{tabularx}

\textbf{Pojašnjenje drugog probnog primjera:}\\

Vrijedi $K = 0$ za prometne planove
\begin{itemize}
\item $\{\}$
\item $\{(1,\,2)\}$
\item $\{(2,\,3)\}$
\end{itemize}

Vrijedi $K = 2$ za prometne planove
\begin{itemize}
\item $\{(1,\,3)\}$
\item $\{(1,\,3),\,(1,\,2)\}$
\item $\{(1,\,3),\,(2,\,3)\}$
\end{itemize}

Vrijedi $K = 3$ za prometne planove
\begin{itemize}
\item $\{(1,\,2),\,(2,\,3)\}$
\item $\{(1,\,2),\,(1,\,3),\,(2,\,3)\}$
\end{itemize}

%%%%%%%%%%%%%%%%%%%%%%%%%%%%%%%%%%%%%%%%%%%%%%%%%%%%%%%%%%%%%%%%%%%%%%
% We're done
\end{statement}

%%% Local Variables:
%%% mode: latex
%%% mode: flyspell
%%% ispell-local-dictionary: "croatian"
%%% TeX-master: "../hio.tex"
%%% End:
