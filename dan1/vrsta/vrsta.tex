%%%%%%%%%%%%%%%%%%%%%%%%%%%%%%%%%%%%%%%%%%%%%%%%%%%%%%%%%%%%%%%%%%%%%%
% Problem statement
\begin{statement}[
  problempoints=100,
  timelimit=1 sekunda,
  memorylimit=512 MiB,
]{Vrsta}\

Mirko ima skrivenu permutaciju $p_1, p_2, \ldots, p_N$ brojeva od $1$ do $N$. Njegov prijatelj Slavko želi otkriti srž Mirkove permutaciju, no on će mu odgovarati samo na pitanja određenog oblika.

Slavko može odabrati bilo koji podniz permutacije, tj.\ segment $p_i, p_{i+1}, \ldots, p_j$ ($1 \le i < j \le N$), i pitati Mirka na kojoj se poziciji nalazi drugi najveći broj u tom segmentu. Mirko mu tada odmah odgovori s traženom pozicijom.

Nakon što mu je odgovorio na sva pitanja, Mirko je odlučio ispitati Slavkovo znanje. Postavit će mu $Q$ upita istoga oblika, a od njega će očekivati da za svaki da točan odgovor.

Slavko ne zna Mirkova pitanja unaprijed, a kako ga ne bi razljutio, želi ga pitati što je moguće manje pitanja. Točnije, Slavko smije postaviti Mirku upit najviše $K$ puta. Pomozite Slavku postaviti pitanja i zatim odgovoriti na Mirkove upite.

%%%%%%%%%%%%%%%%%%%%%%%%%%%%%%%%%%%%%%%%%%%%%%%%%%%%%%%%%%%%%%%%%%%%%%
% Interaction
\subsection*{Interakcija}

Ovo je interaktivni zadatak. Vaš program treba uspostaviti dijalog s programom izrađenim od strane organizatora.

Na početku, vaš program treba sa standardnog ulaza učitati broj $N$, duljinu permutacije.

Zatim može slati upite ispisivanjem na standardni izlaz. Svaki upit mora biti ispisan u zaseban redak i imati oblik "\texttt{?} $i$ $j$", gdje su $i$ i $j$ prirodni brojevi za koje vrijedi $1 \le i < j \le N$. Brojevi $i$ i $j$ predstavljaju granice podniza za koji Slavko želi znati odgovor. Vaš program smije postaviti najviše $K$ ovakvih upita.

Nakon svakog ispisanog upita, program mora napraviti \textit{flush} izlaza te sa standardnog ulaza učitati odgovor na upit — poziciju $k$ za koju vrijedi $i \le k \le j$.

Kada završi s postavljanjem vlastitih upita, program treba ispisati znak "\texttt{!}" kako bi označio kraj Slavkovih pitanja i zatim napraviti \textit{flush} izlaza.

Nakon toga, potrebno je učitati prirodan broj $Q$ — broj Mirkovih upita. Zatim treba učitati $Q$ upita, svaki u obliku "$a$ $b$", gdje su $a$ i $b$ prirodni brojevi za koje vrijedi $1 \le a < b \le N$. Nakon učitavanja svih $Q$ upita, za svaki je potrebno ispisati prirodan broj $k$ — poziciju drugog najvećeg elementa u podnizu $p_a, \ldots, p_b$.

Nakon ispisa odgovora na sve upite, vaš program treba napraviti \textit{flush} izlaza. Kada odgovori na posljednji upit, program može završiti izvođenje.

%%%%%%%%%%%%%%%%%%%%%%%%%%%%%%%%%%%%%%%%%%%%%%%%%%%%%%%%%%%%%%%%%%%%%%
% Scoring
\subsection*{Bodovanje}

%%%%%%%%%	NAPISATI OGRANICENJA

{\renewcommand{\arraystretch}{1.4}
  \setlength{\tabcolsep}{6pt}
  \begin{tabular}{ccl}
   Podzadatak & Broj bodova & Ograničenja \\ \midrule
   	1 & 6 & $N \leq 64$ \\
    2 & 10 & Ne postoji $i$ takav da je $p_i > \max\{p_{i - 1}, p_{i + 1}\}$ \\
    3 & 11 & $p_1 = N$ \\
    4 & 13 & Ne postoji $i$ takav da je $p_i < \min\{p_{i - 1}, p_{i + 1}\}$ \\
    5 & 26 & $N \leq 256$ \\
    6 & 34 & Nema dodatnih ograničenja. \\
\end{tabular}}

%%%%%%%%%%%%%%%%%%%%%%%%%%%%%%%%%%%%%%%%%%%%%%%%%%%%%%%%%%%%%%%%%%%%%%
% Examples
\subsection*{Primjer interakcije}

Pretpostavimo Mirko ima permutatciju \texttt{2 1 4 3}.

{\renewcommand{\arraystretch}{1.4}
  \setlength{\tabcolsep}{6pt}
  \begin{tabular}{lcl}
    Izlaz & Ulaz\\ \midrule
    \texttt{} & \texttt{4} \\
    \texttt{\frenchspacing? 1 2} & \texttt{}  \\
    \texttt{ } & \texttt{2} \\
    \texttt{\frenchspacing? 1 3} & \texttt{} \\
    \texttt{ } & \texttt{1} \\
    \texttt{\frenchspacing!} &\\
    \texttt{} & \texttt{2} \\
    \texttt{} & \texttt{\;\;\;\,1 4} \\
    \texttt{} & \texttt{\;\;\;\,2 3} \\
    \texttt{4} & \texttt{} \\
    \texttt{2} & \texttt{} \\
\end{tabular}}


%%%%%%%%%%%%%%%%%%%%%%%%%%%%%%%%%%%%%%%%%%%%%%%%%%%%%%%%%%%%%%%%%%%%%%
% We're done
\end{statement}

%%% Local Variables:
%%% mode: latex
%%% mode: flyspell
%%% ispell-local-dictionary: "croatian"
%%% TeX-master: "../hio.tex"
%%% End:
