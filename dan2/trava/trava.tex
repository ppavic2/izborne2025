%%%%%%%%%%%%%%%%%%%%%%%%%%%%%%%%%%%%%%%%%%%%%%%%%%%%%%%%%%%%%%%%%%%%%%
% Problem statement
\begin{statement}[
  problempoints=100,
  timelimit=1 sekunda,
  memorylimit=512 MiB,
]{Trava}\

U mirnom kutku grada nalazi se umirovljenički dom čiji stanari vole provoditi vrijeme promatrajući travnjak ispred zgrade. Travnjak je podijeljen na $N$ segmenata, a svaki segment ima visinu trave $a_i$ milimetara, za $1 \leq i \leq N$.

Umirovljenici, zbog godina i dioptrije, ne vide baš savršeno. Kada umirovljenik s dioptrijom $k$ promatra travnjak, on ne razlikuje pojedinačne segmente unutar $k$ uzastopnih dijelova travnjaka. Formalnije, umirovljenik s dioptrijom $k$ na poziciji $i$ vidi visinu trave $\max(a_i, a_{i+1}, \ldots, a_{i+k-1})$ milimetara, za sve $1 \leq i \leq N - k + 1$, dok ostale pozicije ne promatra.

Osim toga, s vremena na vrijeme trava na nekom segmentu može narasti za jedan milimetar, čime se mijenja izgled cijelog travnjaka, a time i visina koje umirovljenici vide.

Potrebno je obraditi $Q$ upita sljedećih oblika:
\vspace{-0.7em}
\begin{itemize}
    \item \texttt{\frenchspacing? k} — umirovljenik s dioptrijom $k$ promatra travnjak. Odredi sumu svih visina koje on vidi.
    \item \texttt{+ i} — trava na $i$-tom segmentu naraste za jedan milimetar.
\end{itemize}

%%%%%%%%%%%%%%%%%%%%%%%%%%%%%%%%%%%%%%%%%%%%%%%%%%%%%%%%%%%%%%%%%%%%%%
% Input
\subsection*{Ulazni podaci}

U prvom retku nalaze se prirodani brojevi $N$ i $Q$ — broj segmenata travnjaka i broj upita.

U drugom retku nalazi se $N$ cijelih brojeva $a_1, a_2, \ldots, a_N$ — početne visine trave.

U idućih $Q$ redaka nalazi se po jedan upit opisan kao:
\vspace{-0.7em}
\begin{itemize}
    \item \texttt{\frenchspacing? k} ($1 \leq k \leq N$)
    \item \texttt{+ i} ($1 \leq i \leq N$)
\end{itemize}

%%%%%%%%%%%%%%%%%%%%%%%%%%%%%%%%%%%%%%%%%%%%%%%%%%%%%%%%%%%%%%%%%%%%%%
% Output
\subsection*{Izlazni podaci}

Za svaki upit tipa \texttt{\frenchspacing? k}, ispiši u zaseban redak jedan cijeli broj — sumu svih visina koje promatra umirovljenik s dioptrijom $k$.

%%%%%%%%%%%%%%%%%%%%%%%%%%%%%%%%%%%%%%%%%%%%%%%%%%%%%%%%%%%%%%%%%%%%%%
% Scoring
\subsection*{Bodovanje}

U svim podzadacima vrijedi $1 \leq N \leq 500\,000$ te $0 \leq Q \leq 500\,000$. Dodatno, za sve $1 \leq i \leq N$ vrijedi $1 \leq A_i \leq 10^9$.

{\renewcommand{\arraystretch}{1.4}
  \setlength{\tabcolsep}{6pt}
  \begin{tabular}{ccl}
   Podzadatak & Broj bodova & Ograničenja \\ \midrule
   	1 & 13 & $N, Q \leq 7\,000$ \\
   	2 & 16 & Ne postoje upiti oblika \texttt{+ i}. \\
   	3 & 23 & U svakom trenutku vrijedit će $A_i \leq 10$ za sve $1 \leq i \leq N$. \\
   	4 & 10 & Vrijedit će da se u svim upitima oblika \texttt{\frenchspacing? k} pojavljuje ista vrijednost $k$. \\
   	5 & 20 & $N, Q \leq 100\,000$ \\
   	6 & 18 & Nema dodatnih ograničenja. \\ 
\end{tabular}}

%%%%%%%%%%%%%%%%%%%%%%%%%%%%%%%%%%%%%%%%%%%%%%%%%%%%%%%%%%%%%%%%%%%%%%
% Examples
\subsection*{Probni primjeri}
\begin{tabularx}{\textwidth}{X'X}
\sampleinputs{test/trava.dummy.in.1}{test/trava.dummy.out.1} &
\sampleinputs{test/trava.dummy.in.2}{test/trava.dummy.out.2}
\end{tabularx}

%%%%%%%%%%%%%%%%%%%%%%%%%%%%%%%%%%%%%%%%%%%%%%%%%%%%%%%%%%%%%%%%%%%%%%
% We're done
\end{statement}

%%% Local Variables:
%%% mode: latex
%%% mode: flyspell
%%% ispell-local-dictionary: "croatian"
%%% TeX-master: "../hio.tex"
%%% End:
