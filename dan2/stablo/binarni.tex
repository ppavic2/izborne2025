%%%%%%%%%%%%%%%%%%%%%%%%%%%%%%%%%%%%%%%%%%%%%%%%%%%%%%%%%%%%%%%%%%%%%%
% Problem statement
\begin{statement}[
  problempoints=100,
  timelimit=1 sekunda,
  memorylimit=512 MiB,
]{Stablo}\

Mirko je od Slavka danas sakrio stablo s $N$ čvorova. Slavka jako zanima kako izgleda Mirkovo stablo, no jedino što zna je da je stupanj svakog čvora unutar stabla najviše $3$ t.j. svaki čvor ima najviše tri susjeda.

Mirko se sažalio nad Slavkom te dopustio mu da postavi $K$ pitanja o stablu. Za danu trojku \textbf{različitih} čvorova $a$, $b$ i $c$, Mirko će mu odgovoriti
\begin{itemize}
\item \texttt{0} ako je udaljenost između čvorova $a$ i $b$ \textbf{jednaka} udaljenosti između čvorova $a$ i $c$
\item \texttt{1} ako je udaljenost između čvorova $a$ i $b$ \textbf{manja} od udaljenosti između čvorova $a$ i $c$
\item \texttt{2} ako je udaljenost između čvorova $a$ i $b$ \textbf{veća} od udaljenosti između čvorova $a$ i $c$.
\end{itemize}

Udaljenost čvorova $u$ i $v$ se definira kao broj bridova na putu između njih.

Pomozite Slavku odrediti Mirkovo stablo!

%%%%%%%%%%%%%%%%%%%%%%%%%%%%%%%%%%%%%%%%%%%%%%%%%%%%%%%%%%%%%%%%%%%%%%
% Interaction
\subsection*{Interakcija}

Ovo je interaktivni zadatak. Vaš program treba uspostaviti dijalog s programom izrađenim od strane organizatora.

Na početku, vaš program treba sa standardnog ulaza učitati broj $N$, veličinu stabla.

Zatim može slati upite ispisivanjem na standardni izlaz. Svaki upit mora biti ispisan u zaseban redak i imati oblik ``\texttt{?} $a$ $b$ $c$'', gdje su $a$, $b$ i $c$ svi prirodni brojevi za koje vrijedi $1 \le a, b, c \le N$ te $a \neq b$, $b \neq c$ i $c \neq a$. Brojevi $a$, $b$ i $c$ predstavljaju čvorove stabla za koje Slavko želi znati odgovor. Vaš program smije postaviti najviše $K$ ovakvih upita.

Nakon svakog ispisanog upita, program mora napraviti \textit{flush} izlaza te sa standardnog ulaza učitati odgovor na upit — nenegativan broj $k$ za koji vrijedi $k \in \{0, 1, 2\}$.

Kada završi s postavljanjem vlastitih upita, program treba ispisati znak ``\texttt{!}'' kako bi označio kraj Slavkovih pitanja i zatim napraviti \textit{flush} izlaza.

Nakon toga, potrebno je ispisati bridove Mirkovog stabla. To jest, ispisati $N - 1$ redaka gdje $i$-ti redak sadrži par brojeva $u_i$ te $v_i$, koji predstavljaju brid u Mirkovom stablu. Nije bitan redoslijed čvorova u bridu niti redoslijed bridova u ispisu. Potrebno je ispisati sve bridove.

Nakon ispisa odgovora, vaš program treba napraviti \textit{flush} izlaza i završiti izvođenje.

%%%%%%%%%%%%%%%%%%%%%%%%%%%%%%%%%%%%%%%%%%%%%%%%%%%%%%%%%%%%%%%%%%%%%%
% Scoring
\subsection*{Bodovanje}

U svim podzadatcima vrijedi $N \leq 512$, $K = 14\,00$.

{\renewcommand{\arraystretch}{1.4}
  \setlength{\tabcolsep}{6pt}
  \begin{tabular}{ccl}
   Podzadatak & Broj bodova & Ograničenja \\ \midrule
   	1 & 6 & $N \leq 64$ \\
    2 & 10 & Ne postoji $i$ takav da je $p_i > \max\{p_{i - 1}, p_{i + 1}\}$ \\
    3 & 11 & $p_1 = N$ \\
    4 & 13 & Ne postoji $i$ takav da je $p_i < \min\{p_{i - 1}, p_{i + 1}\}$ \\
    5 & 26 & $N \leq 256$ \\
    6 & 34 & Nema dodatnih ograničenja. \\
\end{tabular}}

%%%%%%%%%%%%%%%%%%%%%%%%%%%%%%%%%%%%%%%%%%%%%%%%%%%%%%%%%%%%%%%%%%%%%%
% Examples
\subsection*{Primjer interakcije}

Pretpostavimo Mirko ima permutatciju \texttt{2 1 4 3}.

{\renewcommand{\arraystretch}{1.4}
  \setlength{\tabcolsep}{6pt}
  \begin{tabular}{lcl}
    Izlaz & Ulaz\\ \midrule
    \texttt{} & \texttt{4} \\
    \texttt{\frenchspacing? 1 2 3} & \texttt{}  \\
    \texttt{ } & \texttt{1} \\
    \texttt{\frenchspacing? 1 4 3} & \texttt{} \\
    \texttt{ } & \texttt{2} \\
    \texttt{\frenchspacing? 2 1 3} & \texttt{} \\
    \texttt{ } & \texttt{0} \\    
    \texttt{\frenchspacing!} &\\
    \texttt{1 2} & \texttt{} \\
    \texttt{2 3} & \texttt{} \\
    \texttt{3 4} & \texttt{} \\
\end{tabular}}


%%%%%%%%%%%%%%%%%%%%%%%%%%%%%%%%%%%%%%%%%%%%%%%%%%%%%%%%%%%%%%%%%%%%%%
% We're done
\end{statement}

%%% Local Variables:
%%% mode: latex
%%% mode: flyspell
%%% ispell-local-dictionary: "croatian"
%%% TeX-master: "../hio.tex"
%%% End:
