%%%%%%%%%%%%%%%%%%%%%%%%%%%%%%%%%%%%%%%%%%%%%%%%%%%%%%%%%%%%%%%%%%%%%%
% Problem statement
\begin{statement}[
  problempoints=100,
  timelimit=1 sekunda,
  memorylimit=512 MiB,
]{Usisavač}

Mirko ima veliku kuću koja se sastoji od $N$ soba povezanih s $N - 1$ hodnikom. Svaki hodnik povezuje dvije različite sobe i sve su sobe međusobno povezane. Svaki je hodnik dugačak 1 metar. Mirko često čisti sobe u stanu ali rijetko hodnike. U hodnicima se nakupila prašina i Mirko ih sada želi usisati. 

Svaki usisavač ima nažalost i kabel ograničene duljine. Svaka soba ima utičnicu i usisavač mora biti uštekan u utičnicu u nekoj sobi da bi mogao raditi. Mirko kreće od sobe $1$ i može napraviti sljedeće:

Ako usisavač nije uštekan u struju, on može:
\begin{itemize}
	\item Uštekati ga u sobi u kojoj se nalazi. 
	\item Uzeti usisavač u ruke i prijeći u jednu od susjednih soba. Za prolazak hodnikom mu treba 1 minuta.
\end{itemize}

Ako je usisavač uštekan u struju, on može:
\begin{itemize}
	\item Ako se nalazi u sobi u kojoj je uštekao usisavač, može ga odspojiti iz utičnice.
	\item Prijeći u jednu od susjednih hodnika usisavajući hodnik na putu. Ovo može napraviti \textbf{samo ako} je kabel dovoljno dugačak. To jest, ako je udaljenost od sobe u kojoj je uštekan usisavač i ciljane sobe manja ili jednaka od duljine kabla. Za čišćenje hodnika mu treba 1 minuta.
\end{itemize}

Mirkov usisavač se pokvario! Sada je u dućanu u kojem se nalazi $Q$ usisavača, $i$-ti od njih ima duljinu kabla $r_i$ metara. Zanima ga za svaki od usisavača koliko će minimalno trajati usisavanje svih hodnika ako kupi taj usisavač. Pomozite mu odrediti ta vremena!

%%%%%%%%%%%%%%%%%%%%%%%%%%%%%%%%%%%%%%%%%%%%%%%%%%%%%%%%%%%%%%%%%%%%%%
% Input
\subsection*{Ulazni podaci}

U prvom retku su prirodni brojevi $N$ i $Q$, broj soba i broj usisavača.

U idućih $N - 1$ redaka nalaze se prirodni brojevi $x_i$ i $y_i$ $(1 \le x_i, y_i \le N, x_i \ne y_i)$ koji označavaju da postoji hodnik između soba $x_i$ i $y_i$.

U posljednjem retku nalazi se $Q$ brojeva $r_i$ $(1 \le r_i \le N)$, duljine kablova usisavača.

%%%%%%%%%%%%%%%%%%%%%%%%%%%%%%%%%%%%%%%%%%%%%%%%%%%%%%%%%%%%%%%%%%%%%%
% Output
\subsection*{Izlazni podaci}

U jedini redak ispišite $Q$ brojeva gdje $i$-ti broj predstavlja minimalno trajanje čišćenja s $i$-tim usisavačem.

%%%%%%%%%%%%%%%%%%%%%%%%%%%%%%%%%%%%%%%%%%%%%%%%%%%%%%%%%%%%%%%%%%%%%%
% Scoring
\subsection*{Bodovanje}

U svim podzadacima vrijedi $2 \le N \le 3 \cdot 10^5$ i $1 \le Q \le 3 \cdot 10^5$.

{\renewcommand{\arraystretch}{1.4}
  \setlength{\tabcolsep}{6pt}
  \begin{tabular}{ccl}
   Podzadatak & Broj bodova & Ograničenja \\ \midrule
   	1 & 16 & $N, Q \le 1000$ \\
    2 & 10 & Svaka soba $x = 1, 2, \dots, N - 1$ je povezana hodnikom sa sobom $x + 1$. \\
    3 & 22 & $Q = 1$ \\
    4 & 31 & $N, Q \leq 10^5$ \\
    5 & 21 & Nema dodatnih ograničenja. \\
\end{tabular}}

%%%%%%%%%%%%%%%%%%%%%%%%%%%%%%%%%%%%%%%%%%%%%%%%%%%%%%%%%%%%%%%%%%%%%%
% Examples
\subsection*{Probni primjeri}
\begin{tabularx}{\textwidth}{X'X'X}
\sampleinputs{test/usisavac.dummy.in.1}{test/usisavac.dummy.out.1} &
\sampleinputs{test/usisavac.dummy.in.2}{test/usisavac.dummy.out.2} &
\sampleinputs{test/usisavac.dummy.in.3}{test/usisavac.dummy.out.3}
\end{tabularx}

\textbf{Pojašnjenje prvog probnog primjera:}\\

Jedan od načina na koji Mirko može najbrže usisati sve hodnike s duljinom kabla 2m je sljedeći:
\begin{itemize}
	\item Prošeta se od sobe 1 do sobe 3. (2 minute)
	\item Ušteka usisavač u sobi 3.
	\item Usisa hodnike između soba 3 i 4 te 4 i 5 (2 minute).
	\item Vrati se do sobe 3. (2 minute)
	\item Usisa hodnike između soba 3 i 2 te 2 i 1 (2 minute). Time su svi hodnici očišćeni.
\end{itemize}

%%%%%%%%%%%%%%%%%%%%%%%%%%%%%%%%%%%%%%%%%%%%%%%%%%%%%%%%%%%%%%%%%%%%%%
% We're done
\end{statement}

%%% Local Variables:
%%% mode: latex
%%% mode: flyspell
%%% ispell-local-dictionary: "croatian"
%%% TeX-master: "../hio.tex"
%%% End:
